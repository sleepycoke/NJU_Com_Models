\documentclass{article}
\usepackage{ctex}
\usepackage{fontspec, xunicode, xltxtra}
\usepackage{amssymb}
\usepackage{amsmath}
\usepackage{amsfonts}
\usepackage{mathrsfs}
\usepackage{enumerate}
\usepackage{amsthm}
\usepackage{mathtools}
\usepackage{hyperref}
\usepackage[a4paper,top=30mm,bottom=30mm,left=30mm,right=30mm]{geometry}

\newcommand{\vv}[0]{\vdash\dashv}
\renewcommand{\today}{\number\year 年 \number\month 月 \number\day 日}
\newcommand*{\pt}[1]{\ensuremath{\underset{\uparrow}{#1}}}
\newcommand{\RA}[0]{{\shortmid\!\Rightarrow}}
\newcommand{\floor}[1]{\left\lfloor #1 \right\rfloor}
\newcommand{\ceil}[1]{\left\lceil #1 \right\rceil}
\def\dotminus{\mathbin{\ooalign{\hss\raise1ex\hbox{.}\hss\cr
  \mathsurround=0pt$-$}}}
\setlength\parindent{0pt}
\vspace{-8ex}
%\date{}
\begin{document}

\title{《计算模型导引》第五章参考答案\footnote{Ver. 0.2. 原答案由宋方敏教授给出手稿, 然后由丁超录入修订补充. 此文档来源为\url{https://github.com/sleepycoke/NJU_Com_Models}}}
\maketitle

\section*{习题5.1}
主要思想是先后抹除$\overline x$最后移到$\overline y$的头. 
机器如下:
\begin{center}
	
\begin{tabular}{|c|c|c|c|}
	\hline
	&0&1&解释\\
	\hline
	1&0R2&0R1&抹除第一个$\overline x$并指向$\overline y$头\\
	2&0R3&1R2&移动到第二个$\overline x$头\\
	3&0L4&0R3&抹除第二个$\overline x$\\
	4&0L4&1L5&越过$\overline y$的最后一个1\\
	5&0R6&1L5&移到$\overline y$头\\
	\hline
\end{tabular}
\end{center}


\section*{习题5.2}
主要思想是每次抹掉输入的一个1对应两个输出各增加一个1. 
即设计机器$M$实现如下功能:\\
\begin{align*}
	M &| 1 : 0\cdots0\pt101^k01^k0\cdots \twoheadrightarrow u : 0\cdots0\pt1^{k+1}01^{k+1}0\cdots\\
	M &| 1 : 0\cdots0\pt1^{l+1}01^k01^k0\cdots \twoheadrightarrow v : 0\cdots0\pt1^l01^{k+1}01^{k+1}0\cdots\\
\end{align*}
这里$k\ge 0$, $u$为停机状态. 这样$M[v:=1]$既可实现循环. 

具体实现如下:
\begin{center}
\begin{tabular}{|c|c|c|c|}
	\hline
	&0&1&解释\\
	\hline
	1&&0R2&输入移除     1\\
	2&0R3&1R12&判断输入是否还有1,移到尾部再跳过后面的0\\
	3&1R4&1R3&输入为0, 那么走第一个分支3-8.将第一个$1^k$后的0改为1\\
	4&&0R5&将第二个$1^k$的第一个1改为0\\
	5&1R6&1R5&将第二个$1^k$后的0改为1\\
	6&1L7&&再添个1\\
	7&0L8&1L7&跳过$1^{k+1}$\\
	8&0R9&1L8&移到输入头, 停机\\
	12&0R13&1R12&输入不为0, 那么走第二个分支12-19.跳过第一个输入\\
	13&1R14&1R13&将第一个$1^k$后的0改为1\\
	14&&0R15&将第二个$1^k$的第一个1改为0\\
	15&1R16&1R15&将第二个$1^k$后的0改为1\\
	16&1L17&&再添个1\\
	17&0L18&1L17&跳过$1^{k+1}$\\
	18&0R19&1L18&再跳过$1^{k+1}$\\
	19&0R1&1L19&再移到输入头,循环\\
	\hline
\end{tabular}
\end{center}

\section*{习题5.3}
大家可能在某些参考资料上看到了如下的乘法图灵机$M_1$,但它并不适用于我们对输入的定义. 实际上, $M_1$接收的输入为$01^x01^y0\cdots$而不是我们定义的$01^{x+1}01^{y+1}\cdots$. 
\begin{center}
\begin{tabular}{|c|c|c|c|}
	\hline
	&0&1&解释\\
	\hline
	1&0R10&0R2&判断$x$是否为0,不为0抹除一个1进入2-8作一次累加\\
	2&0R3&1R2&移到第一串1的尾部\\
	3&0L8&0R4&3-7作一次累加,作过$y$次后转到8. 进入4之前抹掉1留作标记给7找回用\\
	4&0R5&1R4&找到本串1的尾部\\
	5&1L6&1R5&找到$1^j$的尾部,+1\\
	6&0L7&1L6&找到$1^{j+1}$的头\\
	7&1R3&1L7&找到$1^{i}$的头,补回一个1\\
	8&0L9&1L8&一次累加结束\\
	9&0R1&1L9&找到第一串1的头\\
	10&0R11&0R10&抹除第一串1\\
	\hline
\end{tabular}
\end{center}
我们可以先证明
$$M_1 | 1 : 0\cdots0\pt1^x01^y0\cdots \twoheadrightarrow u : 0\cdots0\pt1^{x*y}0\cdots\\$$
而理解$M1$和关键在于理解3-7状态完成了以下操作:
$$M_1 | 3 : 0\cdots0\pt11^i01^j\cdots \twoheadrightarrow 3 : 1\pt1^i01^{j+1}0\cdots\\$$
从而$M1$和通过3-8状态完成了以下操作:
$$M_1 | 3 : \pt1^i01^j\cdots \twoheadrightarrow 8 : \pt1^i01^{j+i}0\cdots\\$$
即实现了一次累加. 
这样总体上每抹除第一个输入的一个1就,对应一次加法,整体实现了乘法. 


最后为了符合我们的输入输出规范,实现机器$M_2$,$M_3$分别实现:
$$M_2 | 1 : 0\pt1^{x+1}01^{y+1}0\cdots \twoheadrightarrow u : 00\pt1^x01^y0\cdots$$
$$M_3 | 1 : 0\pt1^{x}00\cdots \twoheadrightarrow v : 0\pt1^{x+1}0\cdots$$

这两个机器很容易实现不赘述.

从而且乘法机器为
$$M_2\RA M_1\RA M_3$$

\section*{习题5.4}
 参照第133页定理5.14,我们需要作一个预处理,添上一个输入$\overline 1$. 具体可以由如下机器$M_3$实现:
\begin{center}
\begin{tabular}{|c|c|c|}
	\hline
	&0&1\\
	\hline
	1&0R2&1R1\\
	2&&1R3\\
	3&1L4&\\
	4&&1L5\\
	5&0L6&\\
	6&0R7&1L6
	\\
	\hline
\end{tabular}
\end{center}

如书中第134页构造机器$M_1$:

\begin{center}
\begin{tabular}{|c|c|c|}
	\hline
	&0&1\\
	\hline
	1&&0R2\\
	2&0R$4$&1R3\\
	3&0R4&1R3\\

	
	\hline
\end{tabular}
\end{center}

令$M_2$为$M_1 \RA \fbox{double} \RA \fbox{compress} \RA \fbox{shiftl}$,从而\\
若$x=0$, 则$M_2|1:0\pt{\overline x} 0 \overline y 0 \cdots \twoheadrightarrow v : 000\pt{\overline y} 0 \cdots$\\
若$x>0$, 则$M_2|1:0\pt{\overline x} 0 \overline y 0 \cdots \twoheadrightarrow w : 00{ \overline{\pt x-1}} 0 \overline{g(y)} 0\cdots$, 其中$w$为$M_2$的输出时的状态. \\

令$\fbox{f} = M_2[w:=1]$, $M_3\RA\fbox{f}$即为所求. 


\section*{习题5.5}

\begin{itemize}
	\item 若$x=0$:
	\begin{align*}
		1&:0\pt10000\cdots\tag{1R2}\\
		2&:01\pt0000\cdots\tag{0L3}\\
		3&:0\pt10000\cdots\tag{1L3}\\
		3&:\pt010000\cdots\tag{0L3}\\
		3&:\pt{\phantom0}010000\cdots\tag{halt}\\
	\end{align*}
	\item 若$x>0$:
	\begin{align*}
		1&:0\pt11^x0000\cdots\tag{1R2}\\
		2&:01\pt1^x0000\cdots\tag{0R1}\\
		1&:010\pt1^{x-1}0000\cdots\tag{1R2}\\
		2&:0101\pt1^{x-2}0000\cdots\tag{0R1}\\
		&\cdots \tag{1R2, 0R1}\\
		3&:\overbrace{0101\cdots0\pt1}^{\floor{x/2}+1\text{个01}}0000\cdots\tag{1L3}\\
		&\cdots\tag{0L3, 1L3}\\
		3&:\overbrace{\pt0101\cdots01}^{\floor{x/2}+1\text{个01}}\cdots010000\cdots\tag{0L3}\\
		3&:\pt{\phantom0}\overbrace{0101\cdots01}^{\floor{x/2}+1\text{个01}}\cdots010000\cdots\tag{halt}\\
	\end{align*}
\end{itemize}
总之就是擦除了第偶数个1,然后指向第一个0左边(越界).
\section*{习题5.6}
易见
$M_2|1:0\pt{\overline x} 0 \overline y 0 \cdots \twoheadrightarrow 1 : 000\pt{\overline y} 0 \cdots$\\
从而
$M_2|1:0\pt1^n01^m01^k 0 \overline y 0 \cdots \twoheadrightarrow 1 : 0^{n+m+k+3}\pt0 0 \cdots$(停)\\
又易见
$M_2|1:\pt0 0 0 \cdots \twoheadrightarrow 7 : 0\pt1111 0 \cdots$\\
于是$M_2|1:0\pt1^n01^m01^k 0 \overline y 0 \cdots \twoheadrightarrow
7 : 0^{m+n+k+4}\pt1111 0 \cdots$(停)

\section*{习题5.7}

由于$$y=\floor{\sqrt{x}}\Leftrightarrow y \le \sqrt x < y + 1 \Leftrightarrow y^2 \le x < (y+1)^2$$
所以
$$f(x) = \mu y. x < (y+1)^2 = \mu y. N^2((sx)\dotminus(sy)^2)$$

首先平方机器$\fbox{sqrt}$可由$\fbox{copy1}\RA\fbox{shiftl}\RA\fbox{multi}$构造,其中$\fbox{multi}$为习题5.3的乘法机器. 然后减法机器也在习题5.11中给出,于是$f$涉及到的机器都已经构造出, 那么$\fbox{f}$是可构造的. 具体构造过程可对照教材实现. 

\section*{习题5.10}
参照第133页定理5.14,我们需要作一个预处理,添上一个输入$\overline 0$. 具体可以由如下机器$M_3$实现:
\begin{center}
\begin{tabular}{|c|c|c|}
	\hline
	&0&1\\
	\hline
	1&0R2&1R1\\
	2&1L3&\\
	3&0L4&\\
	4&0R5&1L4
	\\
	\hline
\end{tabular}
\end{center}
接下来再如法构造$\fbox{f}$既可.最后结果为$M_3\RA\fbox f$.
\section*{习题5.11}
注意到$0\dotminus y = 0,\quad x \dotminus 0 = x, \quad (x+1)\dotminus(y+1) = x \dotminus y$, 我们可以分情况处理.
机器如下:
\begin{center}
\begin{tabular}{|c|c|c|c|}
	\hline
	&0&1&解释\\
	\hline
	1&&0R2&将$x$减1\\
	2&0R3&1R4&判断原$x$是否为0\\
	3&1O15&0R3&原$x$为0那么把$\overline y$修改为1输出\\
	4&0R11&1R4&原$x$不为0找到$\overline y$的头\\
	5&0L6&1R5&找到$\overline y$的尾部\\
	6&&0L7&将$y$减1\\
	7&0L10&1L7&找到$\overline x$的尾\\
	10&0R1&1L10&找到$\overline x$的头,迭代\\	
	11&&1R12&跳过$\overline y$的第一个1\\
	12&0L13&1R5&判断$y$是否为0\\
	13&&0L16&原$y$为0,抹掉$\overline y$\\
	14&0R15&1L14&原$y$为0,找到$\overline x$的头,输出$x$\\
	15&&&停机\\
	16&0L14&&前左跳过一个0\\
	\hline
\end{tabular}
\end{center}
计算过程如下:
\begin{itemize}
	\item 若$x=0$:
	\begin{align*}
		1&:0\pt10\overline y 000\cdots\tag{0R2}\\
		2&:00\pt0\overline y 000\cdots\tag{0R3}\\
		3&:000\pt{\overline y}0000\cdots\tag{0R3}\\
		3&:0000{\overline {\pt y - 1}}0000\cdots\tag{0R3}\\
		&\cdots\tag{0R3}\\
		3&:000\cdots\pt00000\cdots\tag{1O15}\\
		15&:000\cdots\pt10000\cdots\tag{halt}
	\end{align*}
	\item 若$x>0,\quad y=0$:
	\begin{align*}
		1&:0\pt11^x01 000\cdots\tag{0R2}\\
		2&:00\pt1^x01000\cdots\tag{1R4}\\
		&\cdots\tag{1R4}\\
		4&:001^{x}\pt01 000\cdots\tag{0R11}\\
		11&:001^{x}0\pt1 000\cdots\tag{1R12}\\
		12&:001^{x}01 \pt000\cdots\tag{0L13}\\
		13&:001^{x}0\pt1 000\cdots\tag{0L16}\\
		16&:001^{x}\pt00 000\cdots\tag{0L14}\\
		14&:001^{x-1}\pt10 000\cdots\tag{1L14}\\
		&\cdots\tag{1L14}\\
		14&:0\pt01^x0 000\cdots\tag{0R15}\\
		15&:00\pt1^x0 000\cdots\tag{halt}\\
	\end{align*}
	\item 若$x>0,\quad y > 0$:
	\begin{align*}
		1&:0\pt11^x0\overline y 000\cdots\tag{0R2}\\
		2&:00\pt1^x0\overline y 000\cdots\tag{1R4}\\
		&\cdots\tag{1R4}\\
		4&:001^{x}\pt0\overline { y} 000\cdots\tag{0R11}\\
		11&:001^{x}0\overline { \pt y} 000\cdots\tag{1R12}\\
		12&:001^{x}01\overline { \pt y-1} 000\cdots\tag{1R5}\\
		&\cdots\tag{1R5}\\
		5&:001^{x}0\overline {y} \pt000\cdots\tag{0L6}\\
		6&:001^x01^{y}\pt1 000\cdots\tag{0L7}\\
		7&:001^x01^{y-1}\pt1 000\cdots\tag{1L7}\\
		&\cdots\tag{1L7}\\
		7&:001^x\pt0\overline { y} 000\cdots\tag{0L10}\\
		10&:001^{x-1}\pt101^{y} 000\cdots\tag{1L10}\\
		&\cdots\tag{1L10}\\
		10&:0\pt01^x01^{y} 000\cdots\tag{0R1}\\
		1&:00\pt1^x01^{y} 000\cdots\tag{Compute $(x-1)\dotminus(y-1)$ }\\
	\end{align*}
\end{itemize}
\section*{习题5.12}
欲证$Even$ Turing-可计算,只需构造机器$E$满足:
\begin{align*}
	E|&1:0\pt1^{2x+1}0\cdots \twoheadrightarrow u : 0\cdots0\pt10\cdots\\
	E|&1:0\pt1^{2x}0\cdots \twoheadrightarrow u : 0\cdots0\pt110\cdots\\
\end{align*}
其中$E(1,u)$无定义. 
$E$可如下构造:
\begin{center}
\begin{tabular}{|c|c|c|}
	\hline
	&0&1\\
	\hline
	1&1L3&0R2\\
	2&1O4&0R4\\
	3&1O3&\\
	\hline
\end{tabular}
\end{center}
$E(1,1)$的$E(1,2)$形成循环将成对的1改为0. 若有奇数个1,最后会执行$M(0,2)$停机,输出$\overline 0$.若有偶数个1,最后会执行$M(0,1),M(0,3)$停机, 输出$\overline 1$. 
\section*{习题5.17}
首先0,1显然都不是一个合法的机器编码, 假定输入至少为2. 

此题的关键在于给定机器$M$的编码$\#M$如何确定其行数.
解法为:

设$M$有$k+1$行,那么显然有$k < \#M$. 
从而设$k = \max z \le \#M, [p_z | \#M]$, 

一但$k$确定, 那么对于各行的解码过程都是可计算的. 一旦出现不合法的行则解码失败, 全部成功解码则说明输入是一个合法的机器编码, 属于$S$. 

反过来由于编码和解码是一对逆过程,那么解码失败的输入一定不属于$S$. 

如果我们承认CT,那么我们已经给出了一个能行的方法判定$S$, 从而$S$是Turing-可计算的. 

若持有更保守的观点, 那可以通过证明$S$为递归集来证明其Turing-可计算. 过程繁杂从略. 

\section*{习题5.18}


由Taylor公式
$$e=\sum_{i=0}^n\frac{1}{i!}+\frac{e^\theta}{(n+1)!}, \quad 0<\theta<1$$
\begin{enumerate}[(1)]
\item 
令$f(n) = \floor{e\cdot n!}$, 我们往证$f \in \mathcal{PRF}$:\\
$f(0)=f(1)=2$, 当$n\ge 2$
$$f(n)=\left(\sum_{i=0}^n\frac{1}{i!}\right)n!+\frac{e^\theta}{n+1}$$
注意到当$n\ge 2$时$0< \frac{e^\theta}{n+1} < 1$, 从而
$$f(n) = \sum_{i=0}^n (n-1)!$$
于是$f \in \mathcal{PRF}$. 
\item
令$h(n)=\floor{e\dot n}$
则$h(0)=0,\quad h(n) = \floor{\frac{g(n)}{(n-1)!}}, n \ge 1$.  \\
从而$h \in \mathcal{PRF}$. 
\item
$$g(1) = 2$$
$$g(n) = h(10^n) \dotminus h(10^{n\dotminus 1}) \cdot 10, \quad n > 1$$
从而$g \in \mathcal{PRF}$, 于是$g$可计算。

\end{enumerate}
注意到余项的上界$\frac{e}{(n+1)!}$严格单调递减且收敛速度超过指数函数, 此题也可以设计数值算法来求$e$的前$n$位有效数字, 进而通过CT说明$g$可计算. 












\end{document}